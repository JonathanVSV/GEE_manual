\usepackage{graphicx}
\usepackage{array}
%%%
%\usepackage{makeidx}
% Para numeración de páginas
\usepackage{fancyhdr}
% Para cuadros de tips
\usepackage{tcolorbox}
\usepackage{awesomebox}
%\usepackage{titling}
% Para fonts, encoding e idioma
\usepackage{lmodern}
\usepackage[utf8]{inputenc}
\usepackage[T1]{fontenc}
\usepackage[spanish, mexico]{babel}
% Agregar guión a palabras que no quepan
\usepackage{hyphenat}
% Título secundario
\usepackage{titlesec}
% Para el control de viudas y huerfanos
\usepackage[defaultlines=4,all]{nowidow}
%\usepackage{flafter}
% Para poner figuras en floats
\usepackage{float}

\usepackage{tocloft}

\usepackage{makeidx}
%\makeindex
%\floatplacement{figure}{btp}

% Definir colores
\definecolor{lightblue}{RGB}{222,243,247}
\definecolor{darkblue}{RGB}{29,77,86}
\definecolor{abvrulecolor}{RGB}{22,92,105}

% Captions de figuras
% Cambiar el color para los títulos de figuras
\usepackage[labelfont={color=darkblue,bf},singlelinecheck=off,center]{caption}

% Re definir títulos y formato
\renewcommand{\chaptermark}[1]{\markboth{#1}{}}
\renewcommand{\sectionmark}[1]{\markright{#1}{}}
\titlelabel{\thetitle\quad}
\titleformat{\chapter}{\normalfont\LARGE\bfseries\color{darkblue}}{\thechapter}{1em}{}
\titleformat*{\section}{\normalfont\Large\bfseries\color{darkblue}}
\titleformat*{\subsection}{\normalfont\large\bfseries\color{darkblue}}
\titleformat*{\subsubsection}{\normalfont\normalsize\bfseries\color{darkblue}}
\titlespacing*{\chapter}{0pt}{2.5ex plus 1ex minus .2ex}{1.3ex plus .2ex}

% Definir cajas de color
\newtcolorbox{bluebox2}{
  colback=lightblue,
  colframe=darkblue,
  coltext=black,
  boxsep=0pt,
  arc=0pt
}

% Formato de página
\renewcommand\headrulewidth{0pt}
%
\pagestyle{empty}
  \fancyhf{}
\fancypagestyle{plain}{}
\fancypagestyle{fancy}{}
%\fancypagestyle{toc}{}
%\fancypagestyle{main}{}
%\fancypagestyle{contents}{}

% Para ayudar a que el float sea menos agresivo
\renewcommand{\topfraction}{.8}
\renewcommand{\bottomfraction}{.8}
\renewcommand{\textfraction}{.05}
\renewcommand{\floatpagefraction}{.75}

% Tratar de cambiar lista de tablas a ejercicios
%\renewcommand{\listtablename}{Lista de Ejercicios}

%\renewcommand{\familydefault}{\rmdefault}
%\renewcommand{\ttdefault}{\ttdefault}

\advance\cftsecnumwidth 0.5em\relax
\advance\cftsubsecindent 0.5em\relax
\advance\cftsubsecnumwidth 0.5em\relax

%\renewcommand\headrulewidth{1pt}
%\pagestyle{fancy} 
%  \fancyfoot{}
%  \fancyfoot[LE,RO]{\thepage}
%  \fancyhead{}
%  \fancyhead[LE,RO]{\leftmark}
%  \fancyhead[LO,RE]{\rightmark}
%\fancypagestyle{plain}{}
\makeindex
%\indexsetup
%\renewcommand{\indexsetup}{firstpagestyle=fancy}